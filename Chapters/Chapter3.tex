% Chapter 3

\chapter{Literature Review}
\label{Chapter3}

Huang et al. (2024) present a novel RTL-to-GDS automation flow specifically designed for adiabatic quantum-flux-parametron (AQFP) superconducting circuits. The authors detail how their custom approach optimizes design tasks at each stage of the flow, including synthesis, placement, and routing, tailored to the unique electrical characteristics of AQFP technologies \cite{huang2024superflow}. Although the focus is on a different circuit type, the core principles of customizing the design flow can be adapted to the UETRV-Pcore project. By analyzing the automation techniques described, the project can enhance its current flow using Cadence Genus and Innovus, potentially leading to better energy efficiency and performance in the RISC-V architecture. This paper serves as a critical reference for exploring custom automation in design flows, encouraging the adaptation of similar strategies for specific project needs.
Acharya and Mehta (2022) conducted a performance analysis comparing the open-source tool Qflow with the commercial tool Cadence Encounter for the RTL to GDS-II flow of a Synchronous FIFO design. They highlighted the accessibility of Qflow for students and researchers, allowing them to engage in projects without the financial burden of expensive commercial tools. Conversely, Cadence Encounter is noted for its efficiency and accuracy, making it a preferred choice in industrial applications. The study revealed that the number of standard cells required when using Qflow was 1.5 times greater than that of Cadence Encounter, resulting in an area requirement over 2.6 times larger. These findings underscore the trade-offs involved in selecting design tools, especially concerning area, power, and operating frequency. This work serves as a critical reference for understanding the implications of tool selection in the RTL to GDS-II process, providing valuable insights that will inform the design decisions for the UET-RV Pcore.

Dwight Hill and Andrew B. Kahng explore the complex journey of chip implementation from RTL (Register Transfer Level) description to GDSII (Graphic Data System II) data, essential for the tape-out process in chip design. They argue that chip implementation encompasses several critical stages, notably logic synthesis, placement, and routing (SP\&R), which have long been supported by advanced commercial tools. These tools have evolved significantly over the years, enabling design teams to refine their approaches through a spiraling methodology that enhances timing estimation, device placement, and accuracy in parasitic extraction. The authors highlight that much of the RTL-to-GDSII work is rooted in industrial practice rather than academic research. This trend is attributed to the complexity of the design process, which requires competitive technology across various software platforms—capabilities that are often beyond the scope of typical graduate projects. They note that the chip design flow involves multiple representations and thus is often referred to as physical synthesis. The process necessitates various libraries, including timing libraries that describe cell delay properties and physical libraries that define the geometry of logic cells and I/O buffers. Additionally, the authors emphasize that the complexity of timing constraints plays a vital role in the design implementation process, reflecting the critical need for timing closure. They discuss the inadequacies of traditional static timing analysis, which is typically conducted at the RTL handoff and mask sign-off milestones. In modern design flows, embedded timing analysis has become integral, driving the need for accurate timing abstractions throughout the entire implementation process \cite{hill2004guest}.
Hill and Kahng address the challenges posed by the non convergence of traditional flows, particularly as designs grow larger and more intricate. They identify various factors, such as crosstalk, substrate coupling, and thermal effects, that influence circuit timing and signal integrity. This complexity necessitates deeper integration between synthesis, analysis, and specification, which the authors argue is crucial for achieving predictable outcomes in chip design. The article presents three main categories of prediction methods aimed at improving design predictability. The first is statistical prediction, which, while quick, often lacks the required accuracy due to its reliance on average metrics. The second is constructive and iterative prediction, which involves real-time estimates based on previous design iterations. Lastly, the authors discuss the importance of enforced assumptions in design properties, which facilitate consistent outcomes throughout the design process \cite{hill2004guest}.
