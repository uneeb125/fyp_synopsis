% Chapter 1

\chapter{Introduction}
\label{Chapter1}

The UETRV-Pcore project is a groundbreaking initiative for Pakistan's academic and industrial sectors. So far, no student team has completed the entire RTL-to-GDS flow using Cadence tools for a complete SoC, which are widely used in the chip design industry. Our project, led by the Department of Electrical Engineering at UET Lahore, aims to fill this gap by developing a complete RISC-V-based application-class System-on-Chip (SoC). This effort is a first of its kind, as we are the pioneers in creating a GDSII (Graphic Data System) layout, the final stage of chip design required for fabrication.
Although Pakistan has a strong foundation in digital design and microarchitecture, the field of physical design and chip fabrication remains largely unexplored. By working on the UETRV-Pcore, we hope to set a new benchmark and encourage engineering programs across the country to focus on physical design. This project will also create opportunities for local chip design innovation and help improve Pakistan’s standing in the global semiconductor industry. If we successfully fabricate the UETRV-Pcore, it will not only validate our team’s expertise but also inspire students, researchers, and industries to explore chip design and fabrication leading to a major shift in the country’s technological landscape.
The UETRV-Pcore is a RISC-V-based application-class System-on-Chip (SoC). It integrates a 32-bit RISC-V ISA core, supporting RV32IMAZicsr instructions, which include base integer operations (I), multiplication and division (M), atomic operations (A), and control/status registers (Zicsr). The core implements three privilege levels—User (U), Supervisor (S), and Machine (M). With instruction and data caches and an SV32-based MMU (Memory Management Unit), UETRV-Pcore is capable of running Linux. The design features peripherals like UART, SPI, CLINT and PLIC that are connected via shared data buses. The design flow for UETRV-Pcore follows a structured physical design methodology using Cadence tools like Genus and Innovus. 
The first step in the physical design flow is synthesis, where the high-level Register Transfer Level (RTL) code is converted into a gate-level netlist. This is done using Cadence Genus, which performs technology mapping by replacing RTL constructs with logic gates from the 45nm standard cell library. Once synthesis is complete, the next step is floorplanning. In this stage, the major blocks of the design such as the core processor, caches, MMU, and peripherals are  placed logically within the chip area. It ensures that sufficient space is allocated for the components while minimizing wire lengths for efficient routing. In the placement stage, the standard cells from the synthesized netlist are placed onto the physical layout according to the floorplan. The tool ensures that the layout has no overlaps and creates space for clock distribution and routing tracks. This stage plays a critical role in improving the timing closure of the design. 
Clock Tree Synthesis (CTS) is one of the most critical stages of the physical design process. The objective of CTS is to create a balanced clock tree that distributes the clock signal uniformly across the chip with minimal clock skew and latency. Clock skew refers to the difference in the arrival time of the clock signal at various points in the design, which can cause timing violations if not controlled. Cadence Innovus uses buffer insertion and gating techniques to build an optimized clock tree. The tool ensures that the clock reaches all registers simultaneously, enabling synchronized operations across the pipeline stages. In the routing stage, the design tool connects the placed cells and peripherals through metal layers to form a complete circuit. During this process, the tool ensures that the design meets electrical rules, timing constraints, and power requirements. Design Rule Checking (DRC) and Layout vs. Schematic (LVS) verification are performed to confirm that the routed design matches the original schematic and complies with the manufacturing rules of the 45nm process.
